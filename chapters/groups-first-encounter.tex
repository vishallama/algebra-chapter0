% Chapter II: Groups, first encounter
\chapter{Groups, first encounter}

\section{Definition of group}
\begin{xca}
Let $(G, \bullet)$ be a group with $e$ denoting the identity element of $G$. We
construct a category $\mc{C}$ as follows:
\begin{itemize}
    \item $\opn{Obj}(\mc{C}) := \{ \ast \}$.
    \item $\Hom_{\mc{C}}(\ast, \ast) := G$.
    \item $1_{\ast} = e$.
\end{itemize}
Then, it is easy to check $\mc{C}$ is indeed a category. Also, since every
element $g \in G$ has an inverse, every morphism $\ast \to \ast$ has an inverse.
That is, every morphism $\ast \to \ast$ is an isomorphism. Thus, $\mc{C}$ is a
groupoid. Hence, we conclude every group is the group of isomorphisms of a
groupoid (with a single object.)

In particular, we note every group is the group of automorphisms of some object
in some category.
\end{xca}

\begin{xca}
$(\Z, +), (\Q, +), (\R, +)$, and $(\C, +)$ are all groups with (additive)
identity $0$.

$(\Q^*, \cdot), (\R^*, \cdot)$, and $(\C^*, \cdot)$ are all groups, with
(multiplicative) identity $1$.
\end{xca}

\begin{xca}
For all elements $g, h$ of a group $G$, we note $(gh)(h^{-1} g^{-1}) =
g(h h^{-1})g^{-1} = g 1_G g^{-1} = g g^{-1} = 1_G$, and $(h^{-1} g^{-1})(gh) =
h^{-1}(g^{-1}g)h = h^{-1} 1_G h = h^{-1} h = 1_G$. Therefore, we conclude
$(gh)^{-1} = h^{-1} g^{-1}$.
\end{xca}

\begin{xca}
Suppose $g^2 = e$ for all elements $g$ of a group $G$. Then, $g^{-1} = g^{-1} e
= g^{-1}(g^2) = (g^{-1} g) g = e g = g$. That is, every element $g$ of $G$ is
its own inverse. Therefore, for all $g, h \in G$, $gh = g^{-1} h^{-1} =
(hg)^{-1} = hg$, thereby implying $G$ is commutative.
\end{xca}

\begin{xca}
Suppose the row corresponding to an element $g$ in the multiplication table of
a group contains the same element in two different columns that correspond to
(distinct) elements, $h_1$ and $h_2$, say. Then, $g h_1 = g h_2$, which, by
cancellation on the left, implies $h_1 = h_2$, a contradiction. \\
The argument is similar for any column of the multiplication table of the group.

Hence, we conclude every row and column of the multiplication table of a group
contains all elements of the group exactly once. \\

The above statements can also be recast in the following form: \\
Let $g$ be an element of a group $G$. Then, the mappings given by \[ h \mapsto
gh \] and \[ h \mapsto hg, \] for all elements $h \in G$, are bijective.
\end{xca}

\begin{xca}
($G$ has one element) This element must be the identity $1_G$. And, thus,
there can be only one multiplication table for $G$: $1_G 1_G = 1_G$.

($G$ has two elements) Let these elements be $1_G$ and $a$. Now, if $a a = a$,
then by the cancellation property, $a = 1_G$, a contradiction. Thus, we must
have $a a = 1_G$. Therefore, there is only one multiplication table for $G$:
$1_G 1_G = 1_G, a a = 1_G$.

\end{xca}
