% Chapter II: Groups, first encounter
\chapter{Groups, first encounter}

\section{Definition of group}
\begin{xca}
Let $(G, \bullet)$ be a group with $e$ denoting the identity element of $G$. We
construct a category $\mc{C}$ as follows:
\begin{itemize}
    \item $\opn{Obj}(\mc{C}) := \{ \ast \}$.
    \item $\Hom_{\mc{C}}(\ast, \ast) := G$.
    \item $1_{\ast} = e$.
\end{itemize}
Then, it is easy to check $\mc{C}$ is indeed a category. Also, since every
element $g \in G$ has an inverse, every morphism $\ast \to \ast$ has an inverse.
That is, every morphism $\ast \to \ast$ is an isomorphism. Thus, $\mc{C}$ is a
groupoid. Hence, we conclude every group is the group of isomorphisms of a
groupoid (with a single object.)

In particular, we note every group is the group of automorphisms of some object
in some category.
\end{xca}

\begin{xca}
$(\Z, +), (\Q, +), (\R, +)$, and $(\C, +)$ are all groups with (additive)
identity $0$.

$(\Q^*, \cdot), (\R^*, \cdot)$, and $(\C^*, \cdot)$ are all groups, with
(multiplicative) identity $1$.
\end{xca}

\begin{xca}
For all elements $g, h$ of a group $G$, we note $(gh)(h^{-1} g^{-1}) =
g(h h^{-1})g^{-1} = g 1_G g^{-1} = g g^{-1} = 1_G$, and $(h^{-1} g^{-1})(gh) =
h^{-1}(g^{-1}g)h = h^{-1} 1_G h = h^{-1} h = 1_G$. Therefore, we conclude
$(gh)^{-1} = h^{-1} g^{-1}$.
\end{xca}

\begin{xca}
Suppose $g^2 = e$ for all elements $g$ of a group $G$. Then, $g^{-1} = g^{-1} e
= g^{-1}(g^2) = (g^{-1} g) g = e g = g$. That is, every element $g$ of $G$ is
its own inverse. Therefore, for all $g, h \in G$, $gh = g^{-1} h^{-1} =
(hg)^{-1} = hg$, thereby implying $G$ is commutative.
\end{xca}

\begin{xca}
Suppose the row corresponding to an element $g$ in the multiplication table of
a group contains the same element in two different columns that correspond to
(distinct) elements, $h_1$ and $h_2$, say. Then, $g h_1 = g h_2$, which, by
cancellation on the left, implies $h_1 = h_2$, a contradiction. \\
The argument is similar for any column of the multiplication table of the group.

Hence, we conclude every row and column of the multiplication table of a group
contains all elements of the group exactly once. \\

The above statements can also be recast in the following form: \\
Let $g$ be an element of a group $G$. Then, the mappings given by \[ h \mapsto
gh \] and \[ h \mapsto hg, \] for all elements $h \in G$, are bijective.
\end{xca}

\begin{xca}
($G$ has one element) This element must be the identity $1_G$. And, thus,
there can be only one multiplication table for $G$: $1_G 1_G = 1_G$.

($G$ has two elements) Let these elements be $1_G$ and $a$. Now, if $a a = a$,
then by the cancellation property, $a = 1_G$, a contradiction. Thus, we must
have $a a = 1_G$. Therefore, there is only one multiplication table for $G$:
$1_G 1_G = 1_G, a a = 1_G$.

($G$ has three elements) Let these distinct elements be $1_G, a, b$. Now, $a b
\neq a$, for otherwise, by cancellation, $b = 1_G$, a contradiction. Similarly,
$ab \neq b$. Thus, we must have $a b = 1_G$. That is, $a$ and $b$ are inverses
of each other. Also, $a a \neq 1$, for otherwise, $a = a^{-1}$, which implies
$a = b$, a contradiction. Again, $a a \neq a$, for otherwise, by cancellation,
$a = 1_G$, a contradiction. Thus, $a a = b$. Similarly, $b b = a$. And, this
exhausts all the possible cases for the multiplication table of $G$. Hence,
there is only one multiplication table for $G$: $a b = 1_G, a a = b, b b = a$.

The above thus shows there is only \emph{one} possible multiplication table for
$G$ if $G$ has exactly 1, 2, or 3 elements.

($G$ has four elements) Let these distinct elements be $1_G, a, b, c$. There
are two possible choices for $ab$: $ab = 1_G$, or $ab = c$. Other choices for
$ab$, \emph{viz.} $a$, or $b$, lead to contradictions. The two choices are:
\begin{itemize}
    \item ($ab = 1_G$) Now, $ac \neq c$, for otherwise, we get $a = 1_G$, a
    contradiction. So, we must have $ac = b$, and thus, $aa = c$. Once we fill
    in the (partially-completed) multiplication table with the above data, we
    can fill in the rest of the multiplication table as follows. We can't have
    $ba = b$, so we must have $ba = 1_G$, and hence, $ca = b$. Then, $bb = c$,
    and $cb = a$, which forces $bc = a$, and $cc = 1_G$.

    The completed multiplication table in this case is as follows:
    \[
    \begin{array}{c || c | c | c | c |}
          & 1 & a & b & c \\
        \hline \hline
        1 & 1 & a & b & c \\
        \hline
        a & a & c & 1 & b \\
        \hline
        b & b & 1 & c & a \\
        \hline
        c & c & b & a & 1 \\
        \hline
    \end{array}
    \]
    A little bit of calculation shows $a^2 = c$, $a^3 = a^2 a = ca = b$, and
    $a^4 = cc = 1_G$. Therefore, the above multiplication table can be entirely
    rewritten in terms of $1_G$ and $a$ as follows:
    \[
    \begin{array}{c || c | c | c | c |}
            & 1   & a   & a^2 & a^3 \\
        \hline \hline
        1   & 1   & a   & a^2 & a^3 \\
        \hline
        a   & a   & a^2 & a^3 & 1 \\
        \hline
        a^2 & a^2 & a^3 & 1   & a \\
        \hline
        a^3 & a^3 & 1   & a   & a^2 \\
        \hline
    \end{array}
    \]
    The above table is precisely the one for the \emph{cyclic group} $C_4$ of
    order $4$.

    \item ($ab = c$) The elements of the group are $1, a, b$, and $ab$. There
    are two possible choices for $a^2$: $1$, or $b$. If $a^2 = b$, then we
    have $b = a^2, ab = a^3$, which reduces to the previous case above. So, we
    are left with the only case to consider: $a^2 = 1$. Then, this forces
    $a(ab) = b$. Now, if $b^2 = a$, then this again reduces to the previous
    case above (up to isomorphism.) So, we are left with only one case to
    consider: $b^2 = 1$. Therefore, $b(ab) = a$. Filling up the rest of the
    multiplication table, we finally obtain $(ab)a = b, (ab)b = a$, and $(ab)^2
    = 1$.

    So, the completed multiplication table looks as follows:
    \[
    \begin{array}{c || c | c | c | c |}
           & 1  & a  & b  & ab \\
        \hline \hline
        1  & 1  & a  & b  & ab \\
        \hline
        a  & a  & 1  & ab & b \\
        \hline
        b  & b  & ab & 1  & a \\
        \hline
        ab & ab & b  & a  & 1 \\
        \hline
    \end{array}
    \]
\end{itemize}
Hence, we conclude there are \emph{two} distinct tables, up to reordering the
elements of $G$, if $G$ has exactly four elements.

Using the tables above, it is also easy to verify all groups with $\le 4$
elements are commutative.
\end{xca}

\begin{xca}
(Prove Corollary 1.11) Let $g$ be an element of finite order, and let $N \in
\Z$. We claim
\[ g^N = e \iff N \text{ is a multiple of } |g|. \]

Note, if $N = 0$, then the statement
\[ g^0 = e \iff 0 \text{ is a multiple of } |g| \]
holds, since both sides of the double implication are trivially true. Hence,
the statement holds for $N = 0$.

We now need prove our claim only for all $N \ne 0$.

($\implies$) Suppose $g^N = e$, where $N \ne 0$. If $N > 0$, then by Lemma 1.10,
$N$ is a multiple of $|g|$. And, if $N < 0$, then $g^{-N} = g^{-N} e = g^{-N}
g^N = g^{-N + N} = g^0 = e$, and since $-N > 0$, again, by Lemma 1.10, $-N$ is
a multiple of $|g|$. Hence, for all $N \ne 0$, $N$ is a multiple of $|g|$.

($\impliedby$) Suppose $N$ is a multiple of $|g|$, where $N \ne 0$. Then, $N =
n |g|$, for some $0 \ne n \in \Z$. Therefore, $g^N = g^{n|g|} = (g^{|g|})^n =
e^n = e$.

Hence, our original claim is proved.
\end{xca}
