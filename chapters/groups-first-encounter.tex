% Chapter II: Groups, first encounter
\chapter{Groups, first encounter}

\section{Definition of group}
\begin{xca}
Let $(G, \bullet)$ be a group with $e$ denoting the identity element of $G$. We
construct a category $\mc{C}$ as follows:
\begin{itemize}
    \item $\opn{Obj}(\mc{C}) := \{ \ast \}$.
    \item $\Hom_{\mc{C}}(\ast, \ast) := G$.
    \item $1_{\ast} = e$.
\end{itemize}
Then, it is easy to check $\mc{C}$ is indeed a category. Also, since every
element $g \in G$ has an inverse, every morphism $\ast \to \ast$ has an inverse.
That is, every morphism $\ast \to \ast$ is an isomorphism. Thus, $\mc{C}$ is a
groupoid. Hence, we conclude every group is the group of isomorphisms of a
groupoid (with a single object.)

In particular, we note every group is the group of automorphisms of some object
in some category.
\end{xca}

\begin{xca}
$(\Z, +), (\Q, +), (\R, +)$, and $(\C, +)$ are all groups with (additive)
identity $0$.

$(\Q^*, \cdot), (\R^*, \cdot)$, and $(\C^*, \cdot)$ are all groups, with
(multiplicative) identity $1$.
\end{xca}

\begin{xca}
For all elements $g, h$ of a group $G$, we note $(gh)(h^{-1} g^{-1}) =
g(h h^{-1})g^{-1} = g 1_G g^{-1} = g g^{-1} = 1_G$, and $(h^{-1} g^{-1})(gh) =
h^{-1}(g^{-1}g)h = h^{-1} 1_G h = h^{-1} h = 1_G$. Therefore, we conclude
$(gh)^{-1} = h^{-1} g^{-1}$.
\end{xca}

\begin{xca}
Suppose $g^2 = e$ for all elements $g$ of a group $G$. Then, $g^{-1} = g^{-1} e
= g^{-1}(g^2) = (g^{-1} g) g = e g = g$. That is, every element $g$ of $G$ is
its own inverse. Therefore, for all $g, h \in G$, $gh = g^{-1} h^{-1} =
(hg)^{-1} = hg$, thereby implying $G$ is commutative.
\end{xca}

\begin{xca}
Suppose the row corresponding to an element $g$ in the multiplication table of
a group contains the same element in two different columns that correspond to
(distinct) elements, $h_1$ and $h_2$, say. Then, $g h_1 = g h_2$, which, by
cancellation on the left, implies $h_1 = h_2$, a contradiction. \\
The argument is similar for any column of the multiplication table of the group.

Hence, we conclude every row and column of the multiplication table of a group
contains all elements of the group exactly once. \\

The above statements can also be recast in the following form: \\
Let $g$ be an element of a group $G$. Then, the mappings given by \[ h \mapsto
gh \] and \[ h \mapsto hg, \] for all elements $h \in G$, are bijective.
\end{xca}

\begin{xca}
($G$ has one element) This element must be the identity $1_G$. And, thus,
there can be only one multiplication table for $G$: $1_G 1_G = 1_G$.

($G$ has two elements) Let these elements be $1_G$ and $a$. Now, if $a a = a$,
then by the cancellation property, $a = 1_G$, a contradiction. Thus, we must
have $a a = 1_G$. Therefore, there is only one multiplication table for $G$:
$1_G 1_G = 1_G, a a = 1_G$.

($G$ has three elements) Let these distinct elements be $1_G, a, b$. Now, $a b
\neq a$, for otherwise, by cancellation, $b = 1_G$, a contradiction. Similarly,
$ab \neq b$. Thus, we must have $a b = 1_G$. That is, $a$ and $b$ are inverses
of each other. Also, $a a \neq 1$, for otherwise, $a = a^{-1}$, which implies
$a = b$, a contradiction. Again, $a a \neq a$, for otherwise, by cancellation,
$a = 1_G$, a contradiction. Thus, $a a = b$. Similarly, $b b = a$. And, this
exhausts all the possible cases for the multiplication table of $G$. Hence,
there is only one multiplication table for $G$: $a b = 1_G, a a = b, b b = a$.

The above thus shows there is only \emph{one} possible multiplication table for
$G$ if $G$ has exactly 1, 2, or 3 elements.

($G$ has four elements) Let these distinct elements be $1_G, a, b, c$. There
are two possible choices for $ab$: $ab = 1_G$, or $ab = c$. Other choices for
$ab$, \emph{viz.} $a$, or $b$, lead to contradictions. The two choices are:
\begin{itemize}
    \item ($ab = 1_G$) Now, $ac \neq c$, for otherwise, we get $a = 1_G$, a
    contradiction. So, we must have $ac = b$, and thus, $aa = c$. Once we fill
    in the (partially-completed) multiplication table with the above data, we
    can fill in the rest of the multiplication table as follows. We can't have
    $ba = b$, so we must have $ba = 1_G$, and hence, $ca = b$. Then, $bb = c$,
    and $cb = a$, which forces $bc = a$, and $cc = 1_G$.

    The completed multiplication table in this case is as follows:
    \[
    \begin{array}{c || c | c | c | c |}
          & 1 & a & b & c \\
        \hline \hline
        1 & 1 & a & b & c \\
        \hline
        a & a & c & 1 & b \\
        \hline
        b & b & 1 & c & a \\
        \hline
        c & c & b & a & 1 \\
        \hline
    \end{array}
    \]
    A little bit of calculation shows $a^2 = c$, $a^3 = a^2 a = ca = b$, and
    $a^4 = cc = 1_G$. Therefore, the above multiplication table can be entirely
    rewritten in terms of $1_G$ and $a$ as follows:
    \[
    \begin{array}{c || c | c | c | c |}
            & 1   & a   & a^2 & a^3 \\
        \hline \hline
        1   & 1   & a   & a^2 & a^3 \\
        \hline
        a   & a   & a^2 & a^3 & 1 \\
        \hline
        a^2 & a^2 & a^3 & 1   & a \\
        \hline
        a^3 & a^3 & 1   & a   & a^2 \\
        \hline
    \end{array}
    \]
    The above table is precisely the one for the \emph{cyclic group} $C_4$ of
    order $4$.

    \item ($ab = c$) The elements of the group are $1, a, b$, and $ab$. There
    are two possible choices for $a^2$: $1$, or $b$. If $a^2 = b$, then we
    have $b = a^2, ab = a^3$, which reduces to the previous case above. So, we
    are left with the only case to consider: $a^2 = 1$. Then, this forces
    $a(ab) = b$. Now, if $b^2 = a$, then this again reduces to the previous
    case above (up to isomorphism.) So, we are left with only one case to
    consider: $b^2 = 1$. Therefore, $b(ab) = a$. Filling up the rest of the
    multiplication table, we finally obtain $(ab)a = b, (ab)b = a$, and $(ab)^2
    = 1$.

    So, the completed multiplication table looks as follows:
    \[
    \begin{array}{c || c | c | c | c |}
           & 1  & a  & b  & ab \\
        \hline \hline
        1  & 1  & a  & b  & ab \\
        \hline
        a  & a  & 1  & ab & b \\
        \hline
        b  & b  & ab & 1  & a \\
        \hline
        ab & ab & b  & a  & 1 \\
        \hline
    \end{array}
    \]
\end{itemize}
Hence, we conclude there are \emph{two} distinct tables, up to reordering the
elements of $G$, if $G$ has exactly four elements.

Using the tables above, it is also easy to verify all groups with $\le 4$
elements are commutative.
\end{xca}

\begin{xca}
(Prove Corollary 1.11) Let $g$ be an element of finite order, and let $N \in
\Z$. We claim
\[ g^N = e \iff N \text{ is a multiple of } |g|. \]

Note, if $N = 0$, then the statement
\[ g^0 = e \iff 0 \text{ is a multiple of } |g| \]
holds, since both sides of the double implication are trivially true. Hence,
the statement holds for $N = 0$.

We now need prove our claim only for all $N \ne 0$.

($\implies$) Suppose $g^N = e$, where $N \ne 0$. If $N > 0$, then by Lemma 1.10,
$N$ is a multiple of $|g|$. And, if $N < 0$, then $g^{-N} = g^{-N} e = g^{-N}
g^N = g^{-N + N} = g^0 = e$, and since $-N > 0$, again, by Lemma 1.10, $-N$ is
a multiple of $|g|$. Hence, for all $N \ne 0$, $N$ is a multiple of $|g|$.

($\impliedby$) Suppose $N$ is a multiple of $|g|$, where $N \ne 0$. Then, $N =
n |g|$, for some $0 \ne n \in \Z$. Therefore, $g^N = g^{n|g|} = (g^{|g|})^n =
e^n = e$.

Hence, our original claim is proved.
\end{xca}

\begin{xca}
Suppose $G$ is a finite abelian group with exactly one element $f$ of order $2$.
That is, $f^2 = 1_G$. Then, $f^{-1} = f$, which means $f$ is its own inverse.
Therefore, every element, other than $f$, of $G$, necessarily has an inverse
that is not $f$. And, since the elements of the group commute, every
(non-identity) term in $\prod_{g \in G} g$ can be placed next to its inverse,
and all these pairs reduce to $1_G$, except $f$. Hence, $\prod_{g \in G} g =
f$.
\end{xca}

\begin{xca}
Let $G$ be a finite group, of order $n$, and let $m$ be the number of elements
$g \in G$ of order exactly $2$. We claim $n-m$ is odd.

First, note the identity element is the only element in a group with order $1$.
Also, it is easy to check, for all non-identity elements $g \in G$,
\[ |g| = 2 \iff g = g^{-1}. \] That is, elements with order $2$ are precisely
the elements that are their own inverses (not counting the identity element.)
This implies elements $g$ with order $> 2$ have inverses that are distinct from
themselves. Thus, these elements come in pairs, which all have distinct
components. So,
\begin{align*}
    |G| &= \sum_{|g| \in \Z^+} \text{ \# of elements with order } |g| \\
        &= \sum_{|g| = 1} \bullet + \sum_{|g| = 2} \bullet + \sum_{|g| > 2}
        \bullet \\
        &= 1 + m + 2k,
\end{align*}
for some $k \in \N$. Therefore, $n - m = 2k + 1$, showing $n-m$ is indeed odd.

In addition, if $n$ is even, then using the previous equation, it is easy to
see $m = n - 2k + 1$ is odd, and hence $G$ necessarily contains elements of
order 2.
\end{xca}

\begin{xca}
Suppose the order of $g$ is odd. Then $|g| = 2n + 1$, for some $n \in \N$.
Using Proposition 1.13, we have
\begin{align*}
    |g^2| &= \frac{\lcm(2, |g|)}{2} \\
          &= \frac{\lcm(2, 2n + 1)}{2} \\
          &= \frac{2(2n + 1)}{2} \\
          &= 2n + 1
\end{align*}
We thus conclude $|g^2| = |g|$.
\end{xca}

\begin{xca}
We claim for all $g, h$ in a group $G$, $|gh| = |hg|$.

To that end, we first prove the following ancillary claim:
\[ |aga^{-1}| = |g| \text{ for all } a, g \in G. \]

Indeed, suppose $a, g$ are elements of a group $G$, and let $|g| = n$, where
$n \in \Z^+$. Then, $(aga^{-1})^n = \underbrace{(aga^{-1}) \cdots
(aga^{-1})}_{n \text{ times}} = a g^n a^{-1} = a 1_G a^{-1} = aa^{-1} = 1_G$.

So, $m = |aga^{-1}|$ divides $n$. Now, suppose $m < n$. Then, $(aga^{-1})^m =
1_G = (aga^{-1})^n$, which implies $a g^m a^{-1} = a g^n a^{-1}$, which by
cancellation, $g^m = g^n$, which implies $g^{n-m} = 1_G$, which (since
$n-m < n$) contradicts the assumption $|g| = n$. Therefore, $|aga^{-1}| = n =
|g|$, which proves our ancillary claim.

Now, using the previous result, we immediately conclude, for all $g, h$ in a
group $G$, $|gh| = |gh1_G| = |gh(gg^{-1})| = |g(hg)g^{-1}| = |hg|$, and we are
done.
\end{xca}

\begin{xca}
In the group of invertible $2 \times 2$ matrices, consider
\begin{center}
    $g =
    \left( \begin{array}{cc}
        0 & -1 \\
        1 & 0 \end{array} \right)$,
    $h =
    \left( \begin{array}{cc}
        0  & 1 \\
        -1 & -1 \end{array} \right)$.
\end{center}
We verify below $|g| = 4, |h| = 3$, and $|gh| = \infty$.

It is easy to verify
\begin{center}
    $g^2 = \left( \begin{array}{cc}
               -1 & 0 \\
                0 & -1 \end{array} \right)$,
    $g^3 = \left( \begin{array}{cc}
        0 & 1 \\
        -1 & 0 \end{array} \right)$, and
    $g^4 = \left( \begin{array}{cc}
        1 & 0 \\
        0 & 1 \end{array} \right)$.
\end{center}
Therefore, $|g| = 4$.

Again, it is easy to verify
\begin{center}
    $h^2 = \left( \begin{array}{cc}
               -1 & -1 \\
                1 & 0 \end{array} \right)$, and
    $h^3 = \left( \begin{array}{cc}
               1 & 0 \\
               0 & 1 \end{array} \right)$.
\end{center}
Therefore, $|h| = 3$. \\

Now, note
\[
gh = \left( \begin{array}{cc}
          1 & 1 \\
          0 & 1 \end{array} \right).
\]
Using induction on $n$, it is easy to show
\[
(gh)^n = \left( \begin{array}{cc}
             1 & n \\
             0 & 1 \end{array} \right),
\]
for all $n \in \Z^+$. Therefore, $(gh)^n \neq I_n$ for any positive integer
$n$, where $I_n$ is the $n \times n$ (identity) diagonal matrix. Hence, $|gh| =
\infty$.
\end{xca}

\begin{xca}
Consider the cyclic group $C_4$ of order $4$, generated by $a$. Choose elements
$g = a$ and $h = a^3$. Then, it is easy to verify $|g| = 4$ and $|h| = 4$.
Therefore, $\lcm(|g|, |h|) = \lcm(4, 4) = 4$. And, $|gh| = |aa^3| = |a^4| =
|e| = 1$. Also, $g$ and $h$ commute. Thus, we have an example wherein $g, h$
commute but $|gh| = 1 \ne 4 = \lcm(|g|, |h|)$.
\end{xca}

\begin{xca}
Suppose $g$ and $h$ commute \emph{and} $\gcd(|g|, |h|) = 1$. We claim $|gh| =
|g||h|$.

To that end, let $m = |g|$ and $n = |h|$, so that $\gcd(m, n) = 1$. By
Proposition 1.14, $|gh|$ divides $\lcm(|g|, |h|) = \lcm(m, n) = mn/\gcd(m, n) =
mn/1 = mn$. Let $N = |gh|$. So, $N \mid mn$.

Now, $(gh)^N = 1$
\begin{align*}
    &\implies g^N = (h^{-1})^N \\
    &\implies |g^N| = |(h^{-1})^N| \\
    &\implies |g^N| = |h^N| \\
    &\implies \frac{|g|}{\gcd(N, |g|)} = \frac{|h|}{\gcd(N, |h|)} \\
    &\implies \frac{m}{\gcd(N, m)} = \frac{n}{\gcd(N, n)} \\
    &\implies m \cdot \gcd(N, n) = n \cdot \gcd(N, m)
\end{align*}
Since $\gcd(m, n) = 1$, it follows $m \mid \gcd(N, m)$ and $n \mid \gcd(N, n)$.
And, since $\gcd(N, m) \mid N$ and $ \gcd(N, n) \mid N$, it follows $m \mid N$
and $n \mid N$, whence $mn \mid N$. But, $N \mid mn$, and so, $N = mn$. Thus,
$|gh| = N = mn = |g||h|$, and this proves our claim.
\end{xca}

\begin{xca}
%TODO: Chapter II - Exercise 1.15
\end{xca}

\section{Examples of groups}
\begin{xca}
One can associate an $n \times n$ matrix $M_{\sigma}$ with a permutation
$\sigma \in S_n$ by letting the entry at $(i, (i)\sigma)$ be $1$ and letting
all other entries be 0. For example, the matrix corresponding to the
permutation
\[
\sigma =
    \begin{pmatrix}
        1 & 2 & 3 \\
        3 & 1 & 2
    \end{pmatrix}
        \in S_3
\]
would be
\[
M_{\sigma} =
    \begin{pmatrix}
        0 & 0 & 1 \\
        1 & 0 & 0 \\
        0 & 1 & 0
    \end{pmatrix}.
\]
We show that, with this notation, \[ M_{\sigma \tau} = M_{\sigma} M_{\tau} \]
for all $\sigma, \tau \in S_n$, where the product on the right is the ordinary
product of matrices.

First, note $\sigma \tau \in S_n$, and so, $M_{\sigma \tau}$ is an $n \times n$
matrix where the entry at $(i, (i)\sigma \tau)$ is $1$ and all other entries is
$0$. And, $M_{\sigma} M_{\tau}$ is an $n \times n$ matrix whose entry at
$(i, j)$ equals \[ \sum_{r=1}^{n} (M_{\sigma})_{i,r} (M_{\tau})_{r,j}, \] which
equals $1$ (otherwise, $0$) iff $(M_{\sigma})_{i,r} = 1 = (M_{\tau})_{r,j}$ for
some $1 \le r \le n$ iff $(i)\sigma = r$ and $(r)\tau = j$ for some $1 \le r \le
n$ iff $((i)\sigma) \tau = j$ iff $(i)(\sigma \tau) = j$ iff the entry at
$(i, (i)\sigma \tau)$ of $M_{\sigma \tau}$ equals $1$ and all other entries is
$0$. We thus conclude $M_{\sigma \tau} = M_{\sigma} M_{\tau}$, and we are done.
\end{xca}

\begin{xca}
Suppose $d \le n$. Then, consider the permutation $\sigma \in S_n$ given by
\[ 1 \to 2 \to 3 \to \ldots \to d \to 1 \] and all positive integers $> d$
being mapped to themselves. Then, clearly, $|\sigma| = d$. This shows $S_n$
contains elements of order $d$.
\end{xca}

\begin{xca}
For every positive integer $n$, an element $\sigma \in S_{\N}$ of order $n$ is
given by the mapping \[ 1 \to 2 \to 3 \to \ldots \to n \to 1 \] such that
$n+i$ is mapped to itself, for all positive integers $i \ge 1$.
\end{xca}

\begin{xca}
%TODO - Chapter II: Ex 2.4
\end{xca}

\begin{xca}
%TODO - Chapter II: Ex 2.5
\end{xca}

\begin{xca}
%TODO - Chapter II: Ex 2.6
\end{xca}

\begin{xca}
%TODO - Chapter II: Ex 2.7
\end{xca}

\begin{xca}
%TODO - Chapter II: Ex 2.8
\end{xca}

\begin{xca}
We verify `congruence mod $n$' is an equivalence relation. Indeed, let $n$ be a
positive integer.

(Reflexivity) For all $a \in \Z$, $n \mid (a - a)$, and so, $a \equiv a \mod n$.

(Symmetry) For all $a, b \in \Z$, if $a \equiv b \mod n$, then $n \mid (b - a)$,
which implies $n \mid (a - b)$, and thus, $b \equiv a \mod n$.

(Transitivity) For all $a, b, c \in \Z$, if $a \equiv b \mod n$ and $b \equiv c
\mod n$, then $n \mid (b - a)$ and $n \mid (c - b)$, and since $c - a = (c - b)
+ (b - a)$, $n \mid (c - a)$, and so, $a \equiv c \mod n$.

And, we are done.
\end{xca}

\begin{xca}
We claim $\Z/{n\Z}$ consists precisely of $n$ elements, for all positive
integers $n$.

Indeed, suppose $n$ is a positive integer. First, note the $n$ equivalence
classes \[ [0]_n, [1]_n, \ldots, [n-1]_n \] are all distinct, for if $[i]_n =
[j]_n$ for some $0 \le i < j < n$, then $n \mid (j-i)$, a contradiction, since
$0 < j-i < n$.

Next, note for any $a \in \Z$, by the Euclidean algorithm, $a = qn + r$, for
some $q, r \in \Z$, where $0 \le r < n$. That is, $n \mid (a-r)$, and so,
$a \equiv r \mod n$, and thus, $[a]_n = [r]_n$. That is to say, the equivalence
class of any integer equals one of the $n$ equivalence classes stated above.

Hence, we conclude $\Z/{n\Z}$ consists precisely of $n$ elements.
\end{xca}

\begin{xca}
We show the square of every odd integer is congruent to $1$ modulo $8$.

Indeed, consider $\Z/{8\Z}$. Suppose $a \in \Z$ is an odd integer. Then, $[a]_8$
equals $[1]_8, [3]_8, [5]_8$, or $[7]_8$. But, $[1]_8^2 = [3]_8^2 = [5]_8^2 =
[7]_8^2 = [1]_8$, which implies $[a]_8^2 = [1]_8$, and hence, $a^2 \equiv 1 \mod
8$, and we are done.
\end{xca}

\begin{xca}
We show there are no \emph{nonzero} integers $a, b, c$ such that $a^2 + b^2 =
3c^2$.

To that end, we study the solutions to the equation $[a]_4^2 + [b]_4^2 =
3[c]_4^2$ in $\Z/{4\Z}$. First, note $[0]_4^2 = [0]_4, [1]_4^2 = [1]_4, [2]_4^2
= [0]_4$, and $[3]_4^2 = [1]_4$. This implies $3[c]_4^2$ equals $[0]_4$ or
$[3]_4$. Thus, any solutions to the equation above exist only when $[a]_4^2 +
[b]_4^2 = [0]_4 = 3[c]_4^2$, and this is possible precisely when $a, b, c$ are
all even.

Now, assume, for the sake of contradiction, triples $(a, b, c)$ (where $a, b,
c$ are all nonzero) exist that are solutions to the original equation $a^2 +
b^2 = 3c^2$. Then, from the foregoing argument, $a, b, c$ must all be even. Let
us restrict our attention, for the moment, only to positive integer solutions
$c$. Using the well-ordering principle, there must exist a smallest $c$, such
that with the corresponding $a$ and $b$, the triple $(a, b, c)$ is a solution
to the original equation. Now, let $a = 2k, b = 2l$, and $c = 2m$, for some
integers $k, l, m$. Therefore, $(2k)^2 + (2l)^2 = 3(2m)^2$, which implies
$k^2 + l^2 = 3m^2$, which is of the same form as the original equation. But,
this implies there exists some triple $(k, l, m)$ that is a solution to the
original equation, where $m = c/2 < c$, thus contradicting the assumption that
$c$ is the smallest positive integer such that $(a, b, c)$ (for some $a, b$) is
a solution to the original equation.

We obtain a similar result if we assume $c$ is a negative integer. Hence, we
conclude there do not exist nonzero integers $a, b, c$, such that $a^2 + b^2 =
3c^2$.
\end{xca}

\begin{xca}
Suppose $\gcd(m, n) = 1$. Then, by Corollary 2.5, the class $[m]_n$ generates
$\Z/{n\Z}$. Therefore, there exists some $a \in \Z$ such that $a[m]_n = [1]_n$,
which implies $[am]_n = [1]_n$, and thus, $am \equiv 1 \mod n$, which implies
$n \mid am - 1$. Therefore, there exists some integer $k$ such that $am - 1 =
kn$, and thus, $am + (-k)n = 1$. We thus conclude there exist integers $a$ and
$b = -k$ such that $am + bn = 1$.

Conversely, suppose $am + bn = 1$ for some integers $a$ and $b$. Then, since
$\gcd(m, n)$ divides both $m$ and $n$, $\gcd(m, n) \mid am + bn = 1$. This
forces $\gcd(m, n) = 1$, and we are done.
\end{xca}

\begin{xca}
(Analog of Lemma 2.2) If $a \equiv a' \mod n$ and $b \equiv b' \mod n$, then
\[ ab \equiv a'b' \mod n. \]

(Proof) Suppose $a \equiv a' \mod n$ and $b \equiv b' \mod n$. Then, $n \mid
a' - a$ and $n \mid b' - b$, and since $a'b' - ab = a'(b' - b) + b(a' - a)$,
$n \mid a'b - ab$, which implies $ab \equiv a'b' \mod n$. And, this completes
our proof.

The above statement shows if $[a]_n = [a']_n$ and $[b]_n = [b']_n$, then
$[ab]_n = [a'b']_n$, thus showing the multiplication on $\Z/{n\Z}$ is a
well-defined operation.
\end{xca}

\begin{xca}
Let $n > 0$ be an odd integer.
\begin{itemize}
\item We show if $\gcd(m, n) = 1$, then $\gcd(2m + n, 2n) = 1$. We prove the
contrapositive of our claim. Indeed, suppose $\gcd(2m + n, 2n) = d \ne 1$. Then,
$d \mid 2m + n$ and $d \mid 2n$. Now, since $n$ is an odd integer, $2m + n$ is
odd and $2n$ is even. Therefore, $d$ must be an odd integer, and since $d$ and
$2$ are relatively prime, $d \mid n$. In addition, $d \mid 2m$, which implies
$d \mid m$, and since $\gcd(m, n) \ge d$, $\gcd(m, n) \ne 1$, and we are done.

\item Suppose $\gcd(r, 2n) = 1$. Then, $r$ must be an odd integer, and hence,
$(r - n)$ is an even integer. Furthermore, there exist integers $a, b$ such that
$ar + b(2n) = 1$, which implies \[ 2a\left(\frac{r - n}{2}\right) + (a + 2b)n =
1. \] Therefore, $\gcd(\frac{r - n}{2}, n) = 1$, and we are done.
\end{itemize}
\end{xca}
