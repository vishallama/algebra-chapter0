% Chapter I: Preliminaries: Set theory and categories
\chapter{Preliminaries: Set theory and categories}

\section{Naive set theory}
\begin{xca}
Let $U = \{ x \mid x \not\in x \}$. Then, $U \not\in U \iff U \in U$, a
contradiction. This is Russell's paradox. Either we assume the \emph{set of all
sets} doesn't exist, or we need to give up the axiom of \emph{unrestricted
comprehension} in set theory.
\end{xca}

\begin{xca}
Suppose $\sim$ is an equivalence relation on a set $S$. For every element $a \in
S$, define the \emph{equivalence class} of $a$ (with respect to $\sim$) by
\begin{center}
$[a]_{\sim} := \{ b \in S \mid b \sim a \}$.
\end{center}
Then, we note that due to \emph{reflexivity}, the equivalence class
$[a]_{\sim}$ of every element $a \in S$ contains $a$, and hence, is nonempty.
Also, $[a]_{\sim} \subset S$, and therefore, $\bigcup_{a \in S} [a]_{\sim} =
S$. Finally, we show the equivalence classes are mutually disjoint. Indeed, for
any two elements $a, b \in S$, if $[a]_{\sim}$ and $[b]_{\sim}$ are disjoint,
then there is nothing to prove. So, suppose $[a]_{\sim} \cap [b]_{\sim}$ is
nonempty. Then, there exists some $c \in S$ that belongs to such an
intersection. Thus, $c \sim a$ and $c \sim b$. By symmetry, $a \sim c$, and
thus, by transitivity, $a \sim b$, which by symmetry again, implies $b \sim a$.
Therefore, for all $x \in [a]_{\sim}$, we have $x \sim a$, and since $a \sim
b$, by transitivity, $x \sim b$, which implies $x \in [b]_{\sim}$, from which
we conclude $[a]_{\sim} \subset [b]_{\sim}$. We can similarly show $[b]_{\sim}
\subset [a]_{\sim}$. Hence, $[a]_{\sim} = [b]_{\sim}$. This establishes
equivalence classes are mutually disjoint. Hence, the set $\mathscr{P}_{\sim}$
of equivalence classes of $S$ is indeed a partition of $S$.
\end{xca}

\begin{xca}
Suppose $\mathscr{P}$ is a partition on a set $S$. Define a relation $\sim$ on
$S$ as follows: For any two elements $a, b \in S$, $a \sim b$ iff $a$ and $b$
belong to the same set in the partition. Then, it is easy to check $\sim$ is
indeed an equivalence relation on $S$. $\mathscr{P}$ is, therefore, the
corresponding partition of the aforesaid equivalence relation, and we are done.
\end{xca}

\begin{xca}
Note the set of equivalence relations on a set $S$ are in a one-to-one
correspondence with the set of partitions of $S$. Thus, the number of different
equivalence relations that may be defined on $S = \{ 1, 2, 3\}$ equals the
number of partitions of $S$, and this number equals $5$, since the partitions
of $S$ are
\begin{center}
    $\{ \{ 1 \}, \{ 2 \}, \{ 3 \} \}, \{ \{ 1, 2 \}, \{ 3 \} \}, \{ \{ 1, 3 \},
    \{ 2 \} \}, \{ \{ 2, 3 \}, \{ 1 \} \}, \{ \{ 1, 2, 3 \} \}$.
\end{center}
The above partitions are also written as $1|2|3, 12|3, 13|2, 23|1, 123$.
\end{xca}
