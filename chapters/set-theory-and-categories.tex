% Chapter I: Preliminaries: Set theory and categories
\chapter{Preliminaries: Set theory and categories}

\section{Naive set theory}
\begin{xca}
Let $U = \{ x \mid x \not\in x \}$. Then, $U \not\in U \iff U \in U$, a
contradiction. This is Russell's paradox. Either we assume the \emph{set of all
sets} doesn't exist, or we need to give up the axiom of \emph{unrestricted
comprehension} in set theory.
\end{xca}

\begin{xca}
Suppose $\sim$ is an equivalence relation on a set $S$. For every element $a \in
S$, define the \emph{equivalence class} of $a$ (with respect to $\sim$) by
\begin{center}
$[a]_{\sim} := \{ b \in S \mid b \sim a \}$.
\end{center}
Then, we note that due to \emph{reflexivity}, the equivalence class
$[a]_{\sim}$ of every element $a \in S$ contains $a$, and hence, is nonempty.
Also, $[a]_{\sim} \subset S$, and therefore, $\bigcup_{a \in S} [a]_{\sim} =
S$. Finally, we show the equivalence classes are mutually disjoint. Indeed, for
any two elements $a, b \in S$, if $[a]_{\sim}$ and $[b]_{\sim}$ are disjoint,
then there is nothing to prove. So, suppose $[a]_{\sim} \cap [b]_{\sim}$ is
nonempty. Then, there exists some $c \in S$ that belongs to such an
intersection. Thus, $c \sim a$ and $c \sim b$. By symmetry, $a \sim c$, and
thus, by transitivity, $a \sim b$, which by symmetry again, implies $b \sim a$.
Therefore, for all $x \in [a]_{\sim}$, we have $x \sim a$, and since $a \sim
b$, by transitivity, $x \sim b$, which implies $x \in [b]_{\sim}$, from which
we conclude $[a]_{\sim} \subset [b]_{\sim}$. We can similarly show $[b]_{\sim}
\subset [a]_{\sim}$. Hence, $[a]_{\sim} = [b]_{\sim}$. This establishes
equivalence classes are mutually disjoint. Hence, the set $\mathscr{P}_{\sim}$
of equivalence classes of $S$ is indeed a partition of $S$.
\end{xca}

\begin{xca}
Suppose $\mathscr{P}$ is a partition on a set $S$. Define a relation $\sim$ on
$S$ as follows: For any two elements $a, b \in S$, $a \sim b$ iff $a$ and $b$
belong to the same set in the partition. Then, it is easy to check $\sim$ is
indeed an equivalence relation on $S$. $\mathscr{P}$ is, therefore, the
corresponding partition of the aforesaid equivalence relation, and we are done.
\end{xca}

\begin{xca}
Note the set of equivalence relations on a set $S$ are in a one-to-one
correspondence with the set of partitions of $S$. Thus, the number of different
equivalence relations that may be defined on $S = \{ 1, 2, 3\}$ equals the
number of partitions of $S$, and this number equals $5$, since the partitions
of $S$ are
\begin{center}
    $\{ \{ 1 \}, \{ 2 \}, \{ 3 \} \}, \{ \{ 1, 2 \}, \{ 3 \} \}, \{ \{ 1, 3 \},
    \{ 2 \} \}, \{ \{ 2, 3 \}, \{ 1 \} \}, \{ \{ 1, 2, 3 \} \}$.
\end{center}
The above partitions are also written as $1|2|3, 12|3, 13|2, 23|1, 123$.
\end{xca}

\begin{xca}
An example of a relation $R$ (defined on a set $S$) that is reflexive and
symmetric but not transitive is the following:
\begin{center}
$R = \{ (1,1), (2,2), (3,3), (1,2), (2,1), (2,3), (3,2) \}$, where
$S = \{ 1,2,3 \}$.
\end{center}
\end{xca}

\begin{xca}
Define a relation $\sim$ on the set $\R$ of real numbers by setting
\begin{center}
 $a \sim b \iff b - a \in \mathbb{Z}$.
\end{center}
We claim $\sim$ is an equivalence relation. To that end, note, for all $a \in
\R$, we have $a \sim a$, since $a - a = 0 \in \Z$. Therefore, $\sim$ is
reflexive. Also, if $a \sim b$, then $b - a \in \Z$, which implies $a - b \in
\Z$, and thus, $b \sim a$. Therefore, $\sim$ is symmetric. Finally, suppose $a
\sim b$ and $b \sim c$. Then, $b - a, c - b \in \Z$, and thus, $c - a = (c - b)
+ (b - a) \in \Z$. Thus, $\sim$ is transitive. Therefore, $\sim$ is an
equivalence relation on $\R$.

(Description of $\sim$) Note all reals that have the same decimal expansion
belong to the same equivalence class under $\sim$. Thus, $[0]_{\sim} = \Z$, and
for any $0 < \alpha < 1, [\alpha]_{\sim} = \{ n + \alpha \mid n \in \Z \}$. This
takes care of all the reals, since each real can always be written as $n +
\alpha$, for some $n \in \Z$ and $0 < \alpha < 1$. Therefore, a `compelling'
description for $\R/{\sim}$ is the unit interval $[0,1]$, such that the
endpoints, $0$ and $1$, are `glued' together. In other words, it is a `loop'
or a 1-sphere.\\

Define a relation $\approx$ on the plane $\R \times \R$ as follows:
\begin{center}
$(a_1, a_2) \approx (b_1, b_2) \iff b_1 - a_1 \in \Z \text{ and }
b_2 - a_2 \in \Z$.
\end{center}
Then, just as above, it is easy to show $\approx$ defines an equivalence
relation on $\R \times \R$. We note $[(0,0)]_{\sim} = \{ (m,n) \mid m,n \in \Z
\}$, and for any $0 < \alpha, \beta < 1, [(\alpha, \beta)]_{\sim} = \{ (m +
\alpha, n + \beta) \mid m,n \in \Z \}$. Thus, a `compelling' description of $\R
\times \R/{\approx}$ is the unit square $[0,1] \times [0,1]$ with the four
corners joined together, so that it forms a 2-sphere.
\end{xca}

\section{Functions between sets}
\begin{xca}
We claim the number of bijections from a set $S$ with $n$ elements to itself is
$n!$. To begin with, any element in $S$ can be mapped to any of the $n$ possible
elements in $S$. Then, the next element in $S$ can be mapped to any of the
remaining $n-1$ elements in $S$, and so on, with the last element in $S$ being
mapped to the last remaining element in $S$. Thus, the number of bijections
equals $n \cdot (n-1) \cdot \ldots \cdot 1 = n!$, which proves our claim.
\end{xca}

\begin{xca}
Assume $A \neq \emptyset$, and let $f: A \to B$ be a function. We claim $f$ has
a right inverse iff it is surjective.

($\impliedby$) Suppose $f$ has a right inverse, $g: B \to A$, say. Then,
$f \circ g = 1_B$. Thus, for all $b \in B$, $b = 1_B(b) = (f \circ g)(b) =
f(g(b)) = f(a)$, where $g(b) = a \in A$. This shows $f$ is surjective.

($\implies$) Suppose $f$ is surjective. Then, for any $b \in B$, the fiber of
$f$ over $b$ is nonempty. Thus, $\{ f^{-1}(b) \}_{b \in B}$ is a family of
nonempty sets, and therefore, using the \emph{axiom of choice}, we can construct
a function $g : B \to A$ as follows: For all $b \in B$, $g(b) = a$ for some
$a \in f^{-1}(b)$. Hence, for all $b \in B, (f \circ g) (b) = f(g(b)) = f(a) =
b = 1_{B} (b)$, and so, $f \circ g = 1_B$. This establishes $g$ is the right
inverse of $f$, and we are done.
\end{xca}

\begin{xca}
Suppose $f : A \to B$ is a bijection. Then, $f$ has an inverse $f^{-1}: B \to A$
such that $f^{-1} \circ f = 1_A$ and $f \circ f^{-1} = 1_B$. Clearly, $f$ is an
inverse of $f^{-1}$, showing $f^{-1}$ is also a bijection.

Suppose $f: A \to B$ and $g: B \to C$ are bijections. We claim $g \circ f: A \to
C$ is also a bijection. To that end, we show $f^{-1} \circ g^{-1}: C \to A$ is
the inverse of $g \circ f$. Indeed, $(g \circ f) \circ (f^{-1} \circ g^{-1}) =
g \circ (f \circ f^{-1}) \circ g^{-1} = g \circ 1_B \circ g^{-1} = g \circ
g^{-1} = 1_C$. And, $(f^{-1} \circ g^{-1}) \circ (g \circ f) = f^{-1} \circ
(g^{-1} \circ g) \circ f = f^{-1} \circ 1_B \circ f = f^{-1} \circ f = 1_A$,
and we are done.
\end{xca}

\begin{xca}
We show `isomorphism' is an equivalence relation on any set of sets.\\
(Reflexivity) For all sets $A, 1_A : A \to A$ is a natural bijection, and thus,
$A \cong A$.\\
(Symmetry) Suppose $A \cong B$ for any two sets $A, B$. Then,
there exists a bijection $f : A \to B$, such that its inverse $f^{-1} : B \to
A$ is also a bijection (as shown in the above exercise.) Thus, $B \cong A$.\\
(Transitivity) Finally, suppose for any three sets, $A, B$ and $C$, $A \cong B$
and $B \cong C$, with $f : A \to B$ and $g: B \to C$ as bijections. Then, from
the previous exercise, $g \circ f : A \to C$ is also a bijection, and thus $A
\cong C$.\\
Thus, our original claim is established.
\end{xca}

\begin{xca}
(\emph{\textbf{Epimorphism}}) A function $f : A \to B$ is an \emph{epimorphism}
(or \emph{epi}) if the following holds: For all sets $Z$ and all functions
$\alpha ', \alpha '' : B \to Z$,
\begin{center}
$\alpha ' \circ f = \alpha '' \circ f \Longrightarrow \alpha ' = \alpha ''$.
\end{center}
In other words, an epimorphism $f$ is \emph{right cancellative}.\\

\emph{Proposition}: A function is surjective iff it is an epimorphism.

\emph{Proof}. ($\implies$) Suppose $f: A \to B$ is an epimorphism. Assume, for
the sake of contradiction, $f$ is \emph{not} surjective. Then, there exists an
element $b_0 \in B$, such that, for all $a \in A$, $f(a) \neq b_0$. We now
construct two distinct functions $\alpha', \alpha'' : B \to \{ 0, 1 \}$ as
follows:
\begin{center}
\[
\alpha'(b) = 0
\]

\[
\alpha''(b) =
    \begin{cases}
        0 & \text{if } b \neq b_0 \\
        1 & \text{if } b = b_0
    \end{cases}
\]
\end{center}
Then, it is easy to check that, for all $a \in A$,
$(\alpha'' \circ f)(a) = \alpha''(f(a)) = 0 = \alpha'(f(a)) = (\alpha' \circ
f)(a)$, which implies $\alpha' \circ f = \alpha'' \circ f$. However, $\alpha'
\neq \alpha''$, which contradicts our assumption that $f$ is an epimorphism.
Hence, we conclude $f$ is surjective.

($\impliedby$) Suppose $f: A \to B$ is surjective. Then, it has a right inverse
$g: B \to A$ such that $f \circ g = 1_B$. Now, assume, for any set $Z$ and any
two functions $\alpha', \alpha'': B \to Z$, $\alpha' \circ f = \alpha'' \circ
f$. Then, $\alpha' = \alpha' \circ 1_B = \alpha' \circ (f \circ g) = (\alpha'
\circ f) \circ g = (\alpha'' \circ f) \circ g = \alpha'' \circ (f \circ g) =
\alpha'' \circ 1_B = \alpha''$, thus proving $f$ is an epimorphism.
\end{xca}

\begin{xca}
Any function $f: A \to B$ determines a section $g: A \to A \times B$ of $\pi_A:
A \times B \to A$ by defining $g$ as follows:
\begin{center}
    $a \mapsto (a, f(a))$
\end{center}
Then, for all $a \in A$, $(\pi_A \circ g)(a) = \pi_A(g(a)) = \pi_A(a, f(a)) =
a = 1_A(a)$, which implies $\pi_A \circ g = 1_A$, thereby showing $g$ as
defined above is indeed a section of $\pi_A$.
\end{xca}

\begin{xca}
Let $f: A \to B$ by any function. We show the graph $\Gamma_f$ of $f$ is
isomorphic to $A$. First, recall the definition of $\Gamma_f$:
\begin{center}
    $\Gamma_f := \{ (a, b) \in (A, B) \mid b = f(a) \} \subseteq A \times B$.
\end{center}
We define a function $g: A \to \Gamma_f$ by
\begin{center}
    $a \mapsto (a, f(a))$.
\end{center}
Then, for any $(a, b) \in \Gamma_f$, we have $b = f(a)$, which implies $g(a) =
(a, f(a)) = (a, b)$, proving $g$ is surjective. Next, for any $a', a'' \in A$,
suppose $g(a') = g(a'')$. This implies $(a', f(a')) = (a'', f(a''))$, which
implies $a' = a''$, thus proving $g$ is injective. Hence, $g$ is an isomorphism,
and so, $A \cong \Gamma_f$.
\end{xca}

\begin{xca}
We describe below explicitly all the terms in the canonical decomposition of the
function $f: \R \to \C$ defined by
\begin{center}
    $r \mapsto e^{2\pi ir}$.
\end{center}
Note $f$ determines an equivalence relation $\sim$ on $\R$ as follows: For all
$r', r'' \in \R$,
\begin{center}
    $r' \sim r'' \iff f(r') = f(r'')$.
\end{center}
Now, $f(r') = f(r'')$ whenever $e^{i2\pi r'} = e^{i2\pi r''}$, \emph{i.e.}
$e^{i2\pi(r' - r'')} = e^0$; that is, $2\pi(r' - r'') = 2\pi k$, where $k \in
\Z$. In other words, $f(r') = f(r'')$ iff $r' - r'' = k$, where $k \in \Z$.
This implies
\begin{center}
    $r' \sim r'' \iff r' - r'' \in \Z$.
\end{center}
The above equivalence relation matches the one stated in Exercise 1.6 of
Chapter 1. And, from the solution to the aforesaid exercise, we note that all
equivalence classes are of the form $[\alpha]_{\sim}$, where $\alpha \in [0,
1)$. Therefore, the isomorphism $\tilde{f}: \R/{\sim} \xrightarrow{\sim}
\opn{im} f$ is defined by
\begin{center}
    $\tilde{f}(\alpha) = e^{2\pi i\alpha}, \, \alpha \in [0, 1)$.
\end{center}
Also, note $\opn{im} f$ is the unit circle on the complex plane. So, the entire
decomposition of $f: \R \to \C$ is as shown below:
\[
\begin{tikzcd}
    \R \arrow[r, two heads]
       \arrow[rrr, bend left, "f"]
       & \R/{\sim} \arrow[r, "\sim"]{r}[swap]{\tilde{f}}
                   & \opn{im} f \arrow[r, hook]
                                & \C
\end{tikzcd},
\]
where $\R/{\sim}$ is a `loop' (1-sphere.)
\end{xca}
